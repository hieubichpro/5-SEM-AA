\chapter{Исследовательская часть}

В данном разделе будут предоставлена информация о технических характеристиках устройства, представлены демонстрации работы программы и проведены замеры процессорного времени.

\section{Технические характеристики}

Технические характеристики устройства, на котором выполнялось тестирование:

\begin{itemize}
	\item[---] операционная система Window 10 Home Single Language;
	\item[---] память 8 Гб;
	\item[---] процессор 11th Gen Intel(R) Core(TM) i7-1165G7 2.80 ГГц, 4 ядра.
\end{itemize}

\section{Демонстрации работы программы}

На рисунках \ref{img:example1}--\ref{img:example2} представлены результаты работы программы.

\img{90mm}{example1}{Демонстрация работы программы при полном переборе}
\img{120mm}{example2}{Демонстрация работы программы при муравьином алгоритме}

\clearpage

\section{Временные характеристики}
В таблице~\ref{tbl:time} приведены результаты замеров по времени рассматриваемых алгоритмов.
\begin{table}[ht]
	\small
	\begin{center}
		\begin{threeparttable}
			\caption{Результаты замеров времени (неотсортированные массивы)}
			\label{tbl:time}
			\begin{tabular}{|c|c|c|}
				\hline
				& \multicolumn{2}{c|}{\bfseries Время, мс} \\ \cline{2-3}
				\bfseries Количество городов & \bfseries Полный перебор & \bfseries Муравьиный алгоритм
				\csvreader{data/time.csv}{} 
				{\\\hline \csvcoli & \csvcolii & \csvcoliii} \\
				\hline
			\end{tabular}	
		\end{threeparttable}
	\end{center}
\end{table}

По таблице~\ref{tbl:time} был построен график, который иллюстрирует зависимость процессорного времени, затраченного реализациями алгоритмов , от количества городов --- рис.~\ref{img:time}.
Исходя из этих данных можно понять, что при количестве городов больше 8 муравьиный алгоритм работает быстрее алгоритма полного перебора.

\img{100mm}{time}{Зависимость времени работы реализаций алгоритмов от размера матрицы}
\clearpage

\section{Автоматическая параметризация}

Автоматическая параметризация была проведена на двух классах данных.
Для проведение эксперимета были взяты матрицы размером 10x10.
Муравьиный алгоритм был запущен для всех значений $\alpha, \rho \in (0, 1)$ с шагом 0.1.

В качестве эталонного значения был взят результат работы алгоритма полного перебора. 

Далее будут представлены матрицы смежности (матрица \ref{eq:kd1} для класса данных, на котором происходила параметризация и таблицы с результами её выполнения.

\subsection{Класс данных}

В качестве класса данных была взята матрица смежности, в которой все значения незначительно отличаются друг от друга, находятся в диапазоне [1, 3].
Таблица с результатами параметризации представлена в приложении А.

\begin{equation}
    \label{eq:kd1}
	M_{1} = \begin{pmatrix}
		0 & 2 & 1 & 1 & 3 & 3 & 2 & 2 & 3 & 3 \\
		2 & 0 & 3 & 3 & 3 & 1 & 3 & 2 & 2 & 3 \\
		1 & 3 & 0 & 2 & 3 & 3 & 1 & 1 & 2 & 3 \\
		1 & 3 & 2 & 0 & 2 & 3 & 2 & 3 & 1 & 3 \\
		3 & 3 & 3 & 2 & 0 & 3 & 2 & 3 & 1 & 1 \\
		3 & 1 & 3 & 3 & 3 & 0 & 1 & 1 & 2 & 1 \\
		2 & 3 & 1 & 2 & 2 & 1 & 0 & 2 & 1 & 2 \\
		2 & 2 & 1 & 3 & 3 & 1 & 2 & 0 & 2 & 3 \\
		3 & 2 & 2 & 1 & 1 & 2 & 1 & 2 & 0 & 2 \\
		3 & 3 & 3 & 3 & 1 & 1 & 2 & 3 & 2 & 0 
	\end{pmatrix}
\end{equation}

\section{Вывод}

В этом разделе были указаны технические характеристики машины, на которой происходило сравнение времени работы алгоритмов (муравьиного алгоритма и алгоритма полного перебора) решения задачи коммивояжера, также была рассмотрена автоматическая параметризация.

В результате замеров времени было установлено, что муравьиный алгоритм работает хуже алгоритма полного перебора на матрицах, размер которых меньше 9.
Но при больших размерах муравьиный алгоритм существенно превосходит алгоритм полного перебора (на матрицах 9x9 он лучше в 2.1 раз, а на матрицах 10x10 уже в 15.4 раза).

На основе проведённой параметризации по двум классам данных можно сделать следующие выводы.

Для класса данных \ref{eq:kd1} лучше всего подходят следующие параметры:
\begin{itemize}[label=---]
    \item $\alpha = 0.1, \beta = 0.9, \rho = 0.2, 0.5$;
    \item $\alpha = 0.2, \beta = 0.8, \rho = 0.3, 0.4, 06$;
    \item $\alpha = 0.3, \beta = 0.7, \rho = 0.2$;
\end{itemize}  

Таким образом, можно сделать вывод о том, что для лучшей работы муравьиного алгоритма на используемых классах данных нужно использовать полученные коэффиценты.

