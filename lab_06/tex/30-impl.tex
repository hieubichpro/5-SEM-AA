\chapter{Технологическая часть}

В данном разделе будут рассмотрены требования к программе, средства реализации, представлены листинги рассматриваемых алгоритмов, а также функциональные тесты.

\section{Требования к программе}
На вход программе должна подаваться матрица стоимостей, которая задает взвешенный неориентированный граф и количество городов.
Выходные данные программы --- оптимальный маршрут, проходящий через все заданные вершины по одному разу с последующим возвратом в исходную точку, и его стоимость.
Программа должна работать в рамках следующих ограничений: 

\begin{itemize}
	\item стоимости путей должны быть целыми числами;
	\item число городов должно быть больше 1;
	\item число дней должно быть больше 0;
\end{itemize}

Пользователь должен иметь возможность выбора метода решения --- полным перебором или муравьиным алгоритмом, и вывода результата на экран.
Кроме того должна быть возможность проведения параметризации муравьиного алгоритма.

\section{Средства реализации}
Для реализации этих алгоритмов был выбран язык программирования $Python$ \cite{python}.
Данный выбор обусловлен наличием у языка функции process\_time измерения процессорного времени.
Визуализация графиков с помощью библиотеки $Matplotlib$ \cite{matplot}.

\section{Реализация алгоритмов}

В листинге \ref{lst:bruteforce} представлена реализация алгоритма полного перебора.
В листингах \ref{lst:antalgo}--\ref{lst:nextcity} представлена реализация муравьиного алгоритма.

\begin{lstinputlisting}[
	caption={Алгоритм полного перебора},
	label={lst:bruteforce},
	linerange={154-173}
	]{../src/algo.py}
\end{lstinputlisting}


\begin{lstinputlisting}[
	caption={Муравьиный алгоритм},
	label={lst:antalgo},
	linerange={101-129}
	]{../src/algo.py}
\end{lstinputlisting}

\begin{lstinputlisting}[
	caption={Функция для обновления феромонов},
	label={lst:pheromone},
	linerange={47-64}
	]{../src/algo.py}
\end{lstinputlisting}

\begin{lstinputlisting}[
	caption={Функция для нахождения вероятней перехода в каждый из городов},
	label={lst:probability},
	linerange={81-98}
	]{../src/algo.py}
\end{lstinputlisting}

\begin{lstinputlisting}[
	caption={Функция выбора следующего города},
	label={lst:nextcity},
	linerange={67-78}
	]{../src/algo.py}
\end{lstinputlisting}


\section{Функциональные тесты}

В таблице \ref{tbl:functional_test} приведены тесты для алгоритмов решения задачи коммивояжера (муравьиного алгоритма и алгоритма полного перебора).
Все тесты пройдены успешно.

\begin{table}[h]
	\begin{center}
        \captionsetup{justification=raggedright,singlelinecheck=off}
		\caption{\label{tbl:functional_test} Функциональные тесты}
		\begin{tabular}{|c|c|c|}
            \hline
        	Входные данные& \multicolumn{2}{c|}{Выходные данные}
        	\\ \hline
        	Матрица смежности & Результат & Ожидаемый результат 
        	\\ \hline
        	$\begin{pmatrix}
        		0 & 1 & 2\\
        		1 & 0 & 3\\
        		2 & 3 & 0
        	\end{pmatrix}$ & 6, [0, 1, 2, 0] & 6, [0, 1, 2, 0]
        	\\ \hline
        	$\begin{pmatrix}
        		0 & 3 & 4 & 1\\
        		3 & 0 & 1 & 1\\
        		4 & 1 & 0 & 2\\
        		1 & 1 & 2 & 0
        	\end{pmatrix}$ & 7, [0, 1, 2, 3, 0] & 7, [0, 1, 2, 3, 0]
        	\\ \hline
        	$\begin{pmatrix}
        		0 & 11 & 12 & 14 & 13\\
        		11 & 0 & 15 & 10 & 10\\
        		12 & 15 & 0 & 14 & 13\\
        		14 & 10 & 14 & 0 & 10\\
        		13 & 10 & 13 & 10 & 0\\
        	\end{pmatrix}$ & 56, [0, 1, 3, 4, 2, 0] & 56, [0, 1, 3, 4, 2, 0]
        	\\ \hline
		\end{tabular}
	\end{center}
\end{table}

\section*{Вывод}
В данном разделе были рассмотрены средства реализации лабораторной работы, представлены реализации алгоритмов, а также функциональные тесты.