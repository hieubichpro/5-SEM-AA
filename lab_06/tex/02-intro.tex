\chapter*{\hfill{\centering ВВЕДЕНИЕ}\hfill}
\addcontentsline{toc}{chapter}{ВВЕДЕНИЕ}
Муравьиный алгоритм — алгоритм для нахождения приближённых решений задач оптимизации на графах, таких, как задача коммивояжера, транспортная задача и аналогичных задач поиска маршрутов на графах.
Суть подхода заключается в анализе и использовании модели поведения муравьёв, ищущих пути от колонии к источнику питания, и представляет собой метаэвристическую оптимизацию.

Целью данной лабораторной работы является изучение задачи коммивояжера, которая решается муравьиным алгоритмом и полным перебором.

Для достижения данной цели необходимо решить следующие задачи:

\begin{itemize}
	\item описать метод полного перебора и муравьиный алгоритм для решения задачи коммивояжёра;
	\item построить схемы рассматриваемых алгоритмов;
	\item создать программное обеспечение, реализующее перечисленные алгоритмы;
	\item провести параметризацию метода, основанного на муравьином алгоритме;
	\item провести сравнительный анализ по времени реализованных алгоритмов;
	\item подготовить отчет о выполненной лабораторной работе.
\end{itemize}