\chapter*{\hfill{\centering ЗАКЛЮЧЕНИЕ}\hfill}
\addcontentsline{toc}{chapter}{ЗАКЛЮЧЕНИЕ}


Было экспериментально подтверждено различие во временной эффективности муравьиного алгоритма и алгоритма полного перебора решения задачи коммивояжера.
В результате исследований можно сделать вывод о том, что при матрицах большого размера (больше 9) стоит использовать муравьиный алгоритм решения задачи коммивояжера, а не алгоритм полного перебора (на матрице размером 10x10 он работает в 15.4 раза быстрее).
Также было установлено по результатам параметризации на экспериментальных класса данных, что при коэффиценте $\alpha$ = 0.1, 0.2, 0.3 муравьиный алгоритм работает наилучшим образом.

Цель, поставленная перед началом работы, была достигнута. В ходе лабораторной работы были решены все задачи:

\begin{itemize}[label = ---]
	\item описаны метод полного перебора и муравьиный алгоритм для решения задачи коммивояжёра;
	\item построены схемы рассматриваемых алгоритмов;
	\item создано программное обеспечение, реализующее перечисленные алгоритмы;
	\item проведена параметризация метода, основанного на муравьином алгоритме;
	\item проведен сравнительный анализ по времени реализованных алгоритмов;
	\item подготовлен отчет о выполненной лабораторной работе.
\end{itemize}
