\chapter{Аналитическая часть}
В данном разделе будет рассмотрена задача коммивояжера и описаны алгоритмы её решения.

\section{Задача коммивояжера}

Цель задачи коммивояжера заключается в нахождении самого выгодного маршрута (кратчайшего, самого быстрого, наиболее дешевого), проходящего через все заданные точки (пункты, города) по одному разу \cite{model}.

Решение методом грубой силы не подходило из-за вычислительной сложности. Была предпринята попытка реализовать метод ветвления и границ с отсечением в глубину. В целом, подход себя оправдывал, но иногда при некоторых специфических входных данных алгоритм выдавал решение далёкое от оптимального.

\section{Алгоритм полного перебора для решения задачи коммивояжера}

Пусть даны $n$ городов и матрица расстояний между ними.
Алгоритм полного перебора для решения задачи коммивояжера предполагает рассмотрение всех возможных путей в графе и выбор наименьшего из них.
Смысл перебора состоит в том, что мы перебираем все варианты объезда городов и выбираем оптимальный.
Однако, при таком подходе количество возможных маршрутов очень быстро возрастает с ростом $n$ (сложность алгоритма равна $n!$).

Алгоритм полного перебора гарантирует точное решение задачи, однако, уже при небольшом числе городов будут большие затраты по времени выполнения.

\section{Муравьиный алгоритм для решения задачи коммивояжера}

Муравьиный алгоритм --- это метаэвристический алгоритм, вдохновленный поведением муравьев при поиске пути к источнику пищи.
Он был успешно применен для решения задачи коммивояжера, которая заключается в поиске кратчайшего пути, проходящего через все города один раз и возвращающегося в исходный город.

Муравьи действуют согласно следующим правилам.
\begin{enumerate}
	\item Муравей запоминает посещенные города, причем каждый город может быть посещен только один раз. Обозначим через $J_{i,k}$ список городов, которые посетил муравей $k$, находящийся в городе $i$.
	\item Муравей обладает видимостью $\eta_{ij}$ - эвристическим желанием посетить город $j$, если муравей находится в городе i, причем
	\begin{equation}
		\label{d_func}
		\eta_{ij} = 1 / D_{ij},
	\end{equation}
	где $D_{ij}$ — стоимость пути из города $i$ в город $j$.
	\item Муравей может улавливать след феромона - специального химического вещества. Число феромона на пути из города $i$ в город $j$ - $\tau_{ij}$.
\end{enumerate}

Муравей выполняет следующую последовательность действий, пока не посетит все города.
\begin{enumerate}
	\item Муравей выбирает следующий город назначения, основываясь на вероятностно - пропорциональном правиле \eqref{posib}, в котором учитываются видимость и число феромона.
	\begin{equation}
		\label{posib}
		P_{ij, k} = \begin{cases}
			\frac{\tau_{ij}^\alpha\eta_{ij}^\beta}{\sum_{l=1}^m \tau^\alpha_{il}\eta^\beta_{il}}, \textrm{если город j необходимо посетить;} \\
			0, \textrm{иначе,}
		\end{cases}
	\end{equation}
	где $\alpha$ - параметр влияния феромона, $\beta$ - параметр влияния видимости пути, $\tau_{ij}$ - число феромона на ребре $(ij)$, $\eta_{ij}$ - эвристическое желание посетить город $j$, если муравей находится в городе $i$.
	Выбор города является вероятностным, данное правило определяет ширину зоны города $j$, в общую зону всех городов $J_{i,k}$ бросается случайное число, которое и определяет выбор муравья.
	\item Муравей проходит путь $(ij)$ и оставляет на нем феромон.
\end{enumerate}

Информация о числе феромона на пути используется другими муравьями для выбора пути. Те муравьи, которые случайно выберут кратчайший путь, будут быстрее его проходить, и за несколько передвижений он будет более обогащен феромоном.
Cледующие муравьи будут предпочитать именно этот путь, продолжая обогащать его феромоном. 

После прохождения маршрутов всеми муравьями значение феромона на путях обновляется в соответствии со следующим правилом \eqref{update_phero_1}.

\begin{equation}
	\label{update_phero_1}
	\tau_{ij}(t+1) = (1-\rho) \tau_{ij}(t) + \Delta \tau_{ij},
\end{equation}
где $\rho$ - коэффициент испарения.
Чтобы найденное локальное решение не было единственным, моделируется испарение феромона.

При этом
\begin{equation}
	\label{update_phero_2}
	\Delta \tau_{ij}(t) = \sum_{k=1}^m \tau_{ij, k}(t),
\end{equation}
где $m$ - число муравьев,
\begin{equation}
	\label{update_phero_3}
	\Delta\tau_{ij,k} = \begin{cases}
		Q/L_{k}, \textrm{если k-ый муравей прошел путь (i,j);} \\
		0,\textrm{иначе.}
	\end{cases}
\end{equation}

\section*{Вывод}

Была описана задача коммивояжёра. Были рассмотрены подходы к решению задачи коммивояжера.