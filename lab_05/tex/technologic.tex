\chapter{Технологическая часть}

В данном разделе будут приведены требования к программному обеспечению, средства реализации, листинги кода, а также функциональные тесты.

\section{Требования к программному обеспечению}

В качестве входных данных задается количество строк и столбцов матрицы $matr$, которое должно быть больше 0, а все элементы матрицы имеют тип $int$. Количество матриц больше 0.
Выходные данные --- табличка с номерами матриц, номерами этапов (лент) её обработки, временем начала обработки текущей матрицы на текущей ленте, временем окончания обработки текущей матрицы на текущей ленте.

\section{Выбор языка программирования}

В данной работе для реализации был выбран язык программирования \textit{C++}~\cite{bib2}, так как он предоставляет весь необходимый функционал для выполнения работы. Для замера времени работы использовалась функция \textit{std::chrono::system\_clock::now()} \cite{bib3}.
Визуализация графиков с помощью библиотеки $Matplotlib$ \cite {bib4}.

\section{Описание используемых типов данных}

При реализации алгоритмов будут использованы следующие структуры данных:
\begin{itemize}[label=---]
	\item матрица - двумерный вектор элементов типа int;
	\item размер матриц и их количество - числа типа int;
\end{itemize}

\section{Реализация алгоритмов}

В листингах \ref{lst:linear}--\ref{lst:lenta3} представлены функции для конвейерного и ленейного алгоритмов обработки матриц.

\begin{lstinputlisting}[
	caption={Алгоритм линейной обработки данных},
	label={lst:linear},
	linerange={217-268}
	]{../src/main/main.cpp}
\end{lstinputlisting}

\begin{lstinputlisting}[
	caption={Алгоритм конвейерной обработки данных},
	label={lst:parallel},
	linerange={339-382}
	]{../src/main/main.cpp}
\end{lstinputlisting}

\begin{lstinputlisting}[
	caption={Алгоритм транспонирования матрицы},
	label={lst:lenta1},
	linerange={30-39}
	]{../src/main/main.cpp}
\end{lstinputlisting}

\begin{lstinputlisting}[
	caption={Алгоритм сложения двух матриц},
	label={lst:lenta2},
	linerange={40-49}
	]{../src/main/main.cpp}
\end{lstinputlisting}

\begin{lstinputlisting}[
	caption={Алгоритм умножения двух матриц},
	label={lst:lenta3},
	linerange={50-60}
	]{../src/main/main.cpp}
\end{lstinputlisting}

\section{Функциональные тесты}

В таблице \ref{tbl:functional_test} приведены функциональные тесты для конвейерного и ленейного алгоритмов обработки матриц. Все тесты пройдены успешно.

\begin{table}[h]
	\begin{center}
	\begin{threeparttable}
		\captionsetup{justification=raggedright,singlelinecheck=off}
		\caption{\label{tbl:functional_test} Функциональные тесты}
		\begin{tabular}{|c|c|c|c|c|}
			\hline
			Строк & Столбцов & Метод обр. & Алгоритм & Ожидаемый результат 
			\\ \hline
			0 & 10 & 10 & Конвейерный & Сообщение об ошибке 
			\\ \hline
			k & 10 & 10 & Конвейерный & Сообщение об ошибке 
			\\ \hline
			10 & 10 & k & Конвейерный & Сообщение об ошибке 
			\\ \hline
			100 & 100 & 20 & Конвейерный & Вывод результ. таблички
			\\ \hline
			100 & 100 & 20 & Линейный & Вывод результ. таблички
			\\ \hline
		\end{tabular}
	\end{threeparttable}
	\end{center}
\end{table}
