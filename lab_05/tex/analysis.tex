\chapter{Аналитическая часть}
В этом разделе будет описан конвейерный принцип обработки данных.

\section{Описание конвейерной обработки данных}

Конвейер — способ организации вычислений, используемый в современных процессорах и контроллерах с целью повышения их производительности (увеличения числа инструкций, выполняемых в единицу времени — эксплуатация параллелизма на уровне инструкций), технология, используемая при разработке компьютеров и других цифровых электронных устройств \cite{bib1}.

Этот способ можно использовать в обработке данных, суть которой состоит в выделении отдельных этапов выполнения общей операции.
Каждый этап, выполнив свою работу, передает результат следующему, одновременно принимая новую порцию данных.

В данной лабораторной работе необходимо реализовать следующую последовательность операций:
\begin{enumerate}
    \item Формулировать матрицу B как сумму матрицы A и транспонированной матрицы A.
    \item Формулировать матрицу C как сумму матрицы B и транспонированной матрицы B.
    \item Заменить матрицу C на произведение C на A.
\end{enumerate}