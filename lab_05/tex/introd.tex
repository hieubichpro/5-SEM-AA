\chapter*{Введение}
\addcontentsline{toc}{chapter}{Введение}

Задачу ускорения обработки данных можно решить с помощью введения конвейерной обработки.
Вводится конвейерная лента и обрабатывающие устройства. Данные поступают на обрабатывающее устройство, которое
после завершения обработки передает их дальше по ленте, и не ожидая завершения цикла, приступает к обработке следующих данных.

Целью данной лабораторной работы является изучение принципов конвейрной обработки данных.

Для достижения поставленной цели необходимо решить следующие задачи:

\begin{itemize}[label=---]
	\item исследовать основы конвейрной обработки данных;
	\item привести схемы алгоритмов, используемых для конвейрной и линейной обработок данных;
	\item реализовать перечисленные алгоритмы;
	\item провести сравнительный анализ времени работы этих алгоритмов;
	\item описать и обосновать полученные результаты в отчете о выполненной лабораторной работе.
\end{itemize}