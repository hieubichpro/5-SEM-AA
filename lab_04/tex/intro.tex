\chapter*{Введение}
\addcontentsline{toc}{chapter}{Введение}

Существуют различные способы написания программ, одним из которых является использование параллельных вычислений.
Параллельные вычисления --- способ организации компьютерных вычислений, при котором программы разрабатываются как набор взаимодействующих вычислительных процессов, работающих параллельно \cite{multi}.

Основная цель параллельных вычислений --- уменьшение времени решения задачи.
Многие задачи требуется решать в реальном времени или для их решения требуется очень большой объем вычислений.
Таким трудоемким алгоритмом является, например, алгоритм Кнута-Морриса-Пратта --- один из алгоритмов поиска подстроки в строке.

Целью данной лабораторной работы является изучение организации параллельных вычислений на базе алгоритма Кнута-Морриса-Пратта.

Для достижения поставленной цели требуется выполнить следующие задачи:

\begin{itemize}
	\item описать последовательный и параллельный алгоритмы поиска подстроки в строке Кнута-Морриса-Пратта;
	\item построить схемы данных алгоритмов;
	\item создать программное обеспечение, реализующее рассматриваемые алгоритмы;
	\item провести сравнительный анализ по времени для реализованного алгоритма;
	\item подготовить отчет о выполненной лабораторной работе.
\end{itemize}