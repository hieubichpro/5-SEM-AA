\chapter{Исследовательская часть}

В данном разделе будут приведены демонстрации работы программы, и будет проведен сравнительный анализ реализованного алгоритма Кнута-Морриса-Пратта по времени.

\section{Технические характеристики}

Тестирование проводилось на устройстве со следующими техническими характеристиками:

\begin{itemize}
	\item операционная система Window 10 Home Single Language;
	\item память 8 Гб;
	\item процессор 11th Gen Intel(R) Core(TM) i7-1165G7 2.80 ГГц, 4 ядра.
\end{itemize}

\section{Демонстрация работы программы}

На рисунках \ref{img:demon1} и \ref{img:demon2} представлен результат работы программы. В каждом примере пользователем введены строка и подстрока и получены все индексы вхождения.

\img{40mm}{demon1}{Демонстрация работы программы при последовательном алгоритме}
\img{40mm}{demon2}{Демонстрация работы программы при параллельном алгоритме}

\section{Время выполнения реализаций алгоритмов}
Результаты замеров времени работы реализаций алгоритмов поиска подстроки в строке преведены в таблице \ref{tbl:4patok}.
Для параллельной реализации алгоритма выбрано количество потоков, равное 4, поскольку именно столько логических ядер имеет процессор ноутбука.
\begin{table}[ht]
	\small
	\begin{center}
		\begin{threeparttable}
			\caption{Результаты замеров времени}
			\label{tbl:4patok}
			\begin{tabular}{|c|c|c|c|}
				\hline
				\bfseries Размер строки & \bfseries Размер подстроки & \bfseries Без многопоточности & \bfseries 4 потока
				\csvreader{csv/4patok.csv}{} 
				{\\\hline \csvcoli & \csvcolii & \csvcoliii & \csvcoliv} \\
				\hline
			\end{tabular}	
		\end{threeparttable}
	\end{center}
\end{table}

По таблице~\ref{tbl:4patok} был построен график, который иллюстрирует зависимость времени, затраченного реализациями алгоритмов КМП, от размера строки --- рис.~\ref{img:4patok}.

\img{100mm}{4patok}{Сравнение времени работы алгоритма без распараллеливания и с 4 вспомогательными потоками при разных размерах строки}
\clearpage
Исходя из этих данных можно понять, что параллельный алгоритм с использованием 4 потоками работает быстрее последовательгого алгоритма КМП.

В таблице~\ref{tbl:manypatok} приведены результаты замеров по времени параллельного алгоритма КМП при разном количество потоков. 

\begin{table}[ht]
	\small
	\begin{center}
		\begin{threeparttable}
			\caption{Результаты замеров времени}
			\label{tbl:manypatok}
			\begin{tabular}{|c|c|}
				\hline
				\bfseries Количество потоков & \bfseries Время выполнения
				\csvreader{csv/manypatok.csv}{} 
				{\\\hline \csvcoli & \csvcolii}\\
				\hline
			\end{tabular}	
		\end{threeparttable}
	\end{center}
\end{table}

По таблице~\ref{tbl:manypatok} был построен график, который иллюстрирует зависимость времени, затраченного реализацией параллельного алгоритма КМП, от количества потоков --- рис.~\ref{img:manypatok}.

\img{100mm}{manypatok}{Сравнение времени работы алгоритма с распараллеливанием на различное количество потоков при размере строки 5000000}

Исходя из этих данных можно понять, параллельный алгоритм работает быстрее всего с 4 потоками.

\section{Вывод}
По графикам видно, что при испольозвании 4 вспомогательных потоков, многопоточная реализация алгоритма значительно эффективнее по времени реализации без многопоточности при работе с строкой разрером 5000000.
Данное количество потоков обусловлено тем, что на ноутбуке, на котором проводились замеры времени, имеется всего 4 логических ядра, а следовательно, количество потоков, при котором потоки будут распределены между всеми ядрами равномерно, равно 4.
Именно поэтому лучшие результаты достигаются именно на 4 потоках.
Исходя из построенных графиков, можно сделать вывод, что распараллеливание кода значительно увеличивает эффективность алгоритма поиска подстроки в строке по времени.