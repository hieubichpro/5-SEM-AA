\chapter{Технологическая часть}

В данном разделе будут указаны средства реализации, будут представлены реализации алгоритмов, а также функциональные тесты.

\section{Требования к программному обеспечению}
Программе, реализующей данные алгоритмы, на вход будут подаваться две строки. Выходным данным такой программы должны быть индексы вхождения подстроки в строке. Программа должна работать в рамках следующих ограничений: 

\begin{itemize}
	\item длины строки и подстроки --- положительное целое число;
	\item длина подстроки не больше длины строки;
	\item должно быть выдано сообщение об ошибке при вводе некорректно.
\end{itemize}


\section{Средства реализации}
Реализация данной лабораторной работы выполнялась при помощи языка программирования С++. Данный выбор обусловлен наличием у языка функции $clock()$ \cite {clock} измерения процессорного времени.

Визуализация графиков с помощью библиотеки $Matplotlib$ \cite {matplot}.

\section{Сведения о модулях программы}

Программа состоит из следующих модулей:

\begin{itemize}
	\item main.cpp --- точка входа программы;
	\item algo.h --- заголовочный файл, содержащий объявлении функций, реализуюших рассматриваемых алгоритмов;
	\item algo.cpp --- файл, содержащий реализации этих функций;
	\item measure.h --- заголовочный файл, содержащий объявлении функций замеров времени работы рассматриваемых алгоритмов;
	\item measure.cpp ---algo файл, содержащий реализации функций замеров времени работы рассматриваемых алгоритмов.
\end{itemize}


\section{Реализация алгоритмов}
Реализации последовательного и параллельного алгоритмов приведены в листингах \ref{lst:bead}--\ref{lst:tree}.

\begin{lstinputlisting}[
	caption={Последовательный алгоритм Кнута-Морриса-Пратта},
	label={lst:bead},
	linerange={30-58}
	]{../src/main/Algo.cpp}
\end{lstinputlisting}

\begin{lstinputlisting}[
	caption={Параллельный алгоритм Кнута-Морриса-Пратта},
	label={lst:radix},
	linerange={96-123}
	]{../src/main/Algo.cpp}
\end{lstinputlisting}

\begin{lstinputlisting}[
	caption={Задача одного потока},
	label={lst:tree},
	linerange={60-94}
	]{../src/main/Algo.cpp}
\end{lstinputlisting}




\section{Функциональные тесты}

В таблице \ref{tbl:tests} приведены функциональные тесты для функции, реализующей алгоритм поиска подстроки в строке. Все тесты пройдены успешно.

\begin{table}[h]
	\begin{center}
		\begin{threeparttable}
			\captionsetup{justification=raggedright,singlelinecheck=off}
			\caption{\label{tbl:tests} Функциональные тесты}
			\begin{tabular}{|c|c|c|c|}
				\hline
				& \multicolumn{2}{c|}{\bfseries Входные данные}& \bfseries Ожидаемый результат \\
				\hline
				№&Строка&Подстрока&Алгоритм КМП \\
				\hline
				1&Пустая строка&Пустая строка&Сообщение об ошибке\\
				\hline
				2&a&aa&Сообщение об ошибке \\
				\hline
				3&abcdabc&bc&[1, 5] \\
				\hline
				4&daymldayaaaymtstadayamkl&day&[0, 5, 17] \\
				\hline
			\end{tabular}
		\end{threeparttable}
	\end{center}
\end{table}

\section*{Вывод}

Были представлены листинги реализаций двух версий алгоритма КМП --- последовательного и параллельного. Также в данном разделе была приведены функциональные тесты.
