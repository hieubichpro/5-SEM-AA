\chapter*{Заключение}
\addcontentsline{toc}{chapter}{Заключение}
При испольозвании 4 вспомогательных потоков, многопоточная реализация алгоритма значительно эффективнее по времени реализации без многопоточности при работе с строкой разрером 5000000.
Данное количество потоков обусловлено тем, что на ноутбуке, на котором проводились замеры времени, имеется всего 4 логических ядра, а следовательно, количество потоков, при котором потоки будут распределены между всеми ядрами равномерно, равно 4.
Именно поэтому лучшие результаты достигаются именно на 4 потоках.
Исходя из построенных графиков, можно сделать вывод, что распараллеливание кода значительно увеличивает эффективность алгоритма поиска подстроки в строке по времени.

Цель лабораторной работы была достигнута, в ходе выполнения лабораторной работы были решены все задачи:
\begin{itemize}
	\item описаны последовательный и параллельный алгоритмы поиска подстроки в строке Кнута-Морриса-Пратта;
	\item построены схемы данных алгоритмов;
	\item создано программное обеспечение, реализующее рассматриваемые алгоритмы;
	\item проведен сравнительный анализ по времени для реализованного алгоритма;
	\item подготовлен отчет о выполненной лабораторной работе.
\end{itemize}
