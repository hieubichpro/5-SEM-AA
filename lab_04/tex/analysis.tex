\chapter{Аналитическая часть}

В данном разделе будут представлены описания последовательного и параллельного вариантов алгоритма Кнута-Морриса-Пратта.

\section{Последовательный алгоритм Кнута-Морриса-Пратта}
Алгоритм Кнута-Морриса-Пратта (KMP) предназначен для эффективного поиска подстроки в строке.
Его ключевая идея заключается в использовании информации о структуре самой подстроки для оптимизации процесса поиска \cite{kmp}.

Принцип его работы следующий:
\begin{itemize}
	\item алгоритм начинает с создания массива lps, где каждый элемент lps[i] представляет собой длину максимального собственного суффикса подстроки s[0:i], который также является её префиксом;
	\item этот массив lps вычисляется для подстроки и используется для определения возможных сдвигов в случае несовпадения символов в процессе поиска;
	\item алгоритм сравнивает символы строки s и подстроки p поочередно;
	\item при несовпадении символов, если мы находимся в середине сравнения, используем значение lps для определения новой позиции в подстроке, с которой продолжим сравнение;
	\item это позволяет избежать бесполезных сравнений, так как мы используем предварительно вычисленные значения lps для оптимизации процесса;
	\item повторяем процесс сравнения до тех пор, пока не найдена подстрока или не достигнут конец строки;
	если найдена подстрока, добавляем индекс её вхождения в результирующий массив.
\end{itemize}

\section{Параллельный алгоритм Кнута-Морриса-Пратта}
Поскольку можно разбить строки на подстроки и обрабатывать их отдельно, то логично назначить каждому потоку часть строки, в которой нужно поискать подстроку \cite{parallel}.
%\img{50mm}{abstract1}{Строка после разбития}

Основная проблема этой версии заключается в том, что если паттерн поступает в часть разделения данных или точку соединения, он не обнаруживается, поскольку данные обрабатываются на разных потоках.
Для решения этой проблемы мы обрабатываем еще часть строки в каждой точке соединения. При этом необходимо использовать 2 мьютекса в теле назначенного потока для организации монопольного доступа к массиву.
%\img{60mm}{abstract2}{Строка до и после присоединения}

\section*{Вывод}

Были изучены алгоритмы поиска подстроки в строке Кнута-Морриса-Пратта и его параллельная версия.