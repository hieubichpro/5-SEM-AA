\chapter{Аналитическая часть}
В этом разделе будет представлено описание алгоритмов поиска целого числа в двоичном дереве поиска несбалансированном и сбалансированном.

\section{Описание двоичного дерева поиска}

Бинарные деревья поиска отличаются от обычных бинарных деревьев тем, что хранят данные в отсортированном виде.
Хранение значений внутри бинарного дерева поиска организовано в следующем виде:
\begin{itemize}[label=---]
	\item все значения в узлах левого дочернего поддерева меньше значения родительского узла;
	\item все значения в узлах правого дочернего поддерева больше значения родительского узла;
	\item каждый дочерний узел тоже является бинарным деревом поиска.
\end{itemize}

Благодаря такой структуре хранения данных поиск узла в бинарном дереве поиска занимает O(log(N)).
Это значительно меньше, если хранить значения в списках — O(N) \cite{bib0}.

\section{Описание сбалансированного двоичного дерева}

Дерево поиска называется сбалансированным, т.е. таким, в котором высота левого и правого поддеревьев отличаются не более чем на единицу \cite{bib1}.

Эта структура данных разработана советскими учеными Адельсон---Вельским Георгием Максимовичем и Ландисом Евгением Михайловичем в 1962 году. 
Аббревиатура АВЛ соответствует первым буквам фамилий этих ученых. 
Первоначально АВЛ-деревья были придуманы для организации перебора в шахматных программах. Советская шахматная программа «Каисса» стала первым официальным чемпионом мира в 1974 году \cite{bib0}.

\section*{Вывод}
В данном разделе приведено описание двоичного дерева поиска и сбалансированного двоичного дерева.
