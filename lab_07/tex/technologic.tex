\chapter{Технологическая часть}

В данном разделе будут приведены средства реализации, реализация алгоритма, а также функциональные тесты.


\section{Средства реализации}

В данной работе для реализации был выбран язык программирования \textit{Python} \cite{bib3}, так как он предоставляет весь необходимый функционал для выполнения работы. 

Визуализация графиков с помощью библиотеки $Matplolib$ \cite{bib4}.

\section{Реализация алгоритмов}

В листингах \ref{lst:algo} представлены функции для алгоритма поиска.

\begin{center}
\captionsetup{justification=raggedright,singlelinecheck=off}
\begin{lstlisting}[label=lst:algo,caption=Функция алгоритма поиска]
def search(self, root, key):
	if root is None or root.key == key:
		return root
	if key < root.key:
		return self.search(root.left, key)
	else:
		return self.search(root.right, key)
\end{lstlisting}
\end{center}

\section{Функциональные тесты}

В таблице \ref{tbl:functional_test} приведены функциональные тесты для двоичного дерева поиска и сбалансированного дерева. Все тесты пройдены успешно.

\begin{table}[h]
	\begin{center}
	\begin{threeparttable}
		\captionsetup{justification=raggedright,singlelinecheck=off}
		\caption{\label{tbl:functional_test} Функциональные тесты}
		\begin{tabular}{|c|c|c|c|}
			\hline
			 N & Элементы & число & Ожидаемый результат 
			\\ \hline
			-1 &  &  & Сообщение об ошибке 
			\\ \hline
			 7 & 1 4 2 3 -1 0 -2 & 0 & Узел в дерево 
			\\ \hline
			 5 & 5 2 3 4 1 & -2 & None 
			\\ \hline
		\end{tabular}
	\end{threeparttable}
	\end{center}
\end{table}

\section*{Вывод}

В данном разделе будут приведены средства реализации, реализация алгоритма, а также функциональные тесты.
