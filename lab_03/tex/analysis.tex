\chapter{Аналитическая часть}
В данном разделе будут описаны алгоритмы сортировок: бисерная сортировка, поразрядная сортировка и сортировка бинарным деревом.

\section{Бисерная сортировка}
\textbf{Бисерная сортировка} --- это уникальный алгоритм сортировки, который использует множество маленьких бусин (шариков), представляющих собой числа или элементы в сортируемом массиве.
Он основан на идее использования гравитации для сортировки \cite{beadsort}.

Принцип работы бисерной сортировки следующий:
\begin{itemize}
	\item создается вертикальный стержень для каждой цифры или элемента в массиве;
	\item каждая бусина помещается на соответствующий стержень, количество бусин на стержне равно значению этой цифры или элемента;
	\item бусины начинают скользить вниз под действием гравитации;
	\item после процесса скольжения бусины распределены по стержням в отсортированном порядке.
\end{itemize}

\section{Поразрядная сортировка}
\textbf{Поразрядная сортировка} --- это алгоритм сортировки, который основывается на разрядах (цифрах) чисел или символов в сортируемом наборе данных \cite{radixsort}.

Её принцип работы заключается в следующем:

\begin{itemize}
	\item первый этап алгоритма начинается с наименьшего разряда (обычно справа) числа и переходит к наибольшему разряду (слева). Для целых чисел это означает, что сначала сортируются единицы, затем десятки, сотни и так далее;
	\item на каждом этапе сортировки элементы разбиваются на группы по значению текущего разряда. Элементы с одинаковым значением разряда помещаются в одну группу;
	\item затем элементы в каждой группе упорядочиваются, обычно с использованием стабильной сортировки, такой как сортировка подсчетом или сортировка вставками;
	\item процесс разбиения на группы и сортировки повторяется для каждого разряда, начиная с наименьшего и заканчивая наибольшим;
	\item после завершения сортировки по всем разрядам, элементы находятся в отсортированном порядке.
\end{itemize}

Поразрядная сортировка обычно применяется к числам или строкам, где разряды представляют собой цифры или символы. Этот алгоритм отличается от сортировок, которые сравнивают элементы напрямую, и может быть эффективным для сортировки больших объемов данных, особенно если разрядность элементов ограничена.

\section{Сортировка бинарным деревом}
\textbf{Сортировка бинарным деревом} --- алгоритм сортировки, который основан на структуре данных двоичного дерева поиска (Binary Search Tree, BST) \cite{treesort}.

Принцип сортировки бинарным деревом:
\begin{itemize}
	\item создается пустое двоичное дерево поиска;
	\item все элементы, которые нужно отсортировать, последовательно вставляются в это дерево. При вставке элемента, он сравнивается с элементами уже находящимися в дереве. Если он меньше, то он помещается в левое поддерево, а если больше то в правое поддерево;
	\item после вставки всех элементов, обход в глубину (in-order traversal) дерева производит элементы в отсортированном порядке. Этот обход следует левому поддереву, затем текущему узлу и, наконец, правому поддереву;
	\item элементы получаются в порядке, обратном порядку их вставки, поэтому для получения отсортированного списка элементов, их нужно сохранить в массив или другую структуру данных.
\end{itemize}

\section*{Вывод}
В данном разделе были теоретически разобраны алгоритмы бисерной сортировки, поразрядной сортировки и сортировки бинарным деревом.