\chapter{Технологическая часть}

В данном разделе будут указаны средства реализации, будут представлены реализации алгоритмов, а также функциональные тесты.

\section{Требования к программному обеспечению}
Программе, реализующей данные алгоритмы, на вход будет подаваться массив. Выходным данным такой программы должна быть такой отсортированный массив. Программа должна работать в рамках следующих ограничений: 

\begin{enumerate}
	\item количесто элементов массива --- положительное целое число;
	\item элементы массива --- натуральные числа;
	\item должно быть выдано сообщение об ошибке при вводе пустого массива.
\end{enumerate}


\section{Средства реализации}
Реализация данной лабораторной работы выполнялась при помощи языка программирования С++. Данный выбор обусловлен наличием у языка функции $clock()$ \cite {clock} измерения процессорного времени.

Визуализация графиков с помощью библиотеки $Matplotlib$ \cite {matplot}.

\section{Сведения о модулях программы}

Программа состоит из следующих модулей:

\begin{itemize}
	\item main.cpp --- точка входа программы;
	\item algo.h --- заголовочный файл, содержащий объявлении функций, реализуюших рассматриваемых алгоритмов;
	\item algo.cpp --- файл, содержащий реализации этих функций;
	\item measure.h --- заголовочный файл, содержащий объявлении функций замеров времени работы и затравоченной памяти рассматриваемых алгоритмов;
	\item measure.cpp ---algo файл, содержащий реализации функций замеров времени работы и затравоченной памяти рассматриваемых алгоритмов.
\end{itemize}


\section{Реализация алгоритмов}
Алгоритм бисерной сортировки, алгоритм поразрядной сортировки и алгоритм сортировки бинарным деревом приведены в листингах \ref{lst:bead}--\ref{lst:tree}.

\begin{lstinputlisting}[
	caption={Алгоритм бисерной сортировки},
	label={lst:bead},
	linerange={29-48}
	]{../src/main/Algo.cpp}
\end{lstinputlisting}

\begin{lstinputlisting}[
	caption={Алгоритм поразрядной сортировки},
	label={lst:radix},
	linerange={50-72}
	]{../src/main/Algo.cpp}
\end{lstinputlisting}

\begin{lstinputlisting}[
	caption={Алгоритм сортировки бинарным деревом},
	label={lst:tree},
	linerange={88-119}
	]{../src/main/Algo.cpp}
\end{lstinputlisting}

\clearpage




\section{Функциональные тесты}

В таблице \ref{tbl:tests} приведены функциональные тесты для функций, реализующих алгоритмы сортировок. Все тесты пройдены успешно.

\begin{table}[h]
	\begin{center}
        \begin{threeparttable}
        \captionsetup{justification=raggedright,singlelinecheck=off}
		\caption{\label{tbl:tests} Функциональные тесты}
		\begin{tabular}{|c|c|c|c|c|}
			\hline
			&Входные данные& \multicolumn{3}{c|}{Ожидаемый результат} \\
			\hline
			№&Массив&Бисерная&Поразрядрая&Бинарным деревом \\
			\hline
            1&[ ]&[ ]&[ ]&[ ] \\
            \hline
            2&[1]&[1]&[1]&[1] \\
            \hline
            3&[2, 2, 2, 2, 2]&[2, 2, 2, 2, 2]&[2, 2, 2, 2, 2]&[2, 2, 2, 2, 2] \\
            \hline
            4&[1, 2, 3, 4]&[1, 2, 3, 4]&[1, 2, 3, 4]&[1, 2, 3, 4] \\
			\hline
			5&[8, 6, 3, 1]&[8, 6, 3, 1]&[8, 6, 3, 1]&[8, 6, 3, 1] \\
			\hline
            6&[8, 1, 12, 4]&[1, 4, 8, 12]&[1, 4, 8, 12]&[1, 4, 8, 12] \\
			\hline
		\end{tabular}
        \end{threeparttable}
	\end{center}
\end{table}

\section*{Вывод}

Были представлены листинги реализаций всех алгоритмов сортировок --- бисерной, поразрядной и бинарного дерева. Также в данном разделе была приведены функциональные тесты.