\chapter*{Введение}
\addcontentsline{toc}{chapter}{Введение}

При решении различных задач встает необходимость работы с упорядоченным набором данных. Например, при поиске элемента в заданном множестве.
Для упорядочивания последовательности значений используется сортировка.

\textbf{Сортировка} --- процесс перегруппировки заданного множества объектов в некотором определенном порядке.
Для реализации этого процесса разрабатываются алгоритмы сортировки \cite{def}. 

Существует большое количество алгоритмов сортировки.
Все они решают одну и ту же задачу, причем некоторые алгоритмы имеют преимущества перед другими.
Поэтому существует необходимость сравнительного анализа алгоритмов сортировки. 

Целью данной лабораторной работы является описание трудоемкости алгоритмов сортировки.
Для решения поставленной цели требуется решить следующие задачи:

\begin{itemize}
	\item описать алгоритмы бисерной сортировки, поразрядной сортировки, сортировки бинарным деревом;
	\item построить схемы рассматриваемых алгоритмов;
	\item создать программное обеспечение, реализующее перечисленные алгоритмы;
	\item провести сравнительный анализ реализованных алгоритмов по времени и по памяти;
	\item подготовить отчет о выполненной лабораторной работе.
\end{itemize}
