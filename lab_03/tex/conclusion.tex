\chapter*{Заключение}
\addcontentsline{toc}{chapter}{Заключение}

%В результате исследования было определено, что при увеличении длины входного массива время работы реализаций алгоритмов сортировки увеличивается по разным законам.

%Трудоемкости:
%\begin{itemize}[label=---]
%	\item бисерная сортировка -- O($NM$);
%	\item поразрядная сортировка -- O($N\log(M)$);
%	\item сортровка бинарным деревом -- O($N\log(N)$) в лучшем случае, O($N^2$) в худшем случае.
%\end{itemize}

В результате исследования было получено, что при размере массива больше 500, поразрядная сортировка работает быстрее сортировки бинарным деревом в 20 раза.
При этом бисерная сортировка работает медленнее сортировки бинарным деревом в 8,9 раза.
В случае, если массив уже отсортирован, время выполнения сортировки бинарным деревом в 1.3 раза увеличивается из-за несбалансированного дерева.

При размере массива больше 500 бисерная сортировка использует памяти в 4,5 раза больше остальных.
При этом поразрядная сортировка и сортировка бинарным деревом используют примерно одинаковое количество памяти.
Можно сделать вывод, что поразрядная сортировка предпочтительна использовать для сортировки массива.

Цель, которая была поставлена в начале лабораторной работы, была достигнута.
В ходе лабораторной работы решены все задачи:
\begin{itemize}
	\item описаны алгоритмы бисерной сортировки, поразрядной сортировки, сортировки бинарным деревом;
	\item построены схемы рассматриваемых алгоритмов;
	\item создано программное обеспечение, реализующее перечисленные алгоритмы;
	\item проведен сравнительный анализ реализованных алгоритмов по времени и по памяти;
	\item подготовлен отчет о выполненной лабораторной работе.
\end{itemize}