\chapter{Технологическая часть}

В данном разделе будут указаны средства реализации, будут представлены реализация алгоритма, а также функциональные тесты.



\section{Средства реализации}

Данная программа разработана на языке C++, поддерживаемом многими операционными системами \cite{bib2}.



\section{Реализация алгоритмов}

Aлгоритм поиска элемент в массиве приведен в листинге \ref{lst:solution}.

\begin{lstinputlisting}[
	caption={Алгоритм поиска элемента в массиве},
	label={lst:solution},
	linerange={5-15}
	]{../src/main.cpp}
\end{lstinputlisting}

\clearpage
 
\section{Функциональные тесты}

В таблице \ref{tbl:tests} приведены функциональные тесты для функций, реализующих алгоритмы. Все тесты пройдены успешно.

\begin{table}[h]
	\begin{center}
		\begin{threeparttable}
			\captionsetup{justification=raggedright,singlelinecheck=off}
			\caption{\label{tbl:tests} Функциональные тесты}
			\begin{tabular}{|c|c|c|}
				\hline
				№&Входные данные& Ожидаемый результат \\
				\hline
				1&1 2 3 4 5, 10& -1\\
				\hline
				2&1 2 3 4 5, 5& 4\\
				\hline
				3&1 2 3 4 5, 1& 0\\
				\hline
			\end{tabular}
		\end{threeparttable}
	\end{center}
\end{table}

\section*{Вывод}
Были реализованы функции поиска полным перебором.
Было проведено функциональное тестирование данного алгоритма.