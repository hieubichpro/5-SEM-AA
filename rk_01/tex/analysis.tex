\chapter{Аналитическая часть}

\section{Трудоёмкость в худшем и лучшем случаях}
Под худшим случаем будем понимать такой вход D длины n, на котором алгоритм A задает наибольшее количество элементарных операций, при этом максимум берется по всем $D \in D_{n}$. Трудоемкость алгоритма на этом входе будем называть трудоемкостью в худшем случае, и обозначать ее через ${f^\wedge_{A}}(n)$, тогда
\begin{equation}
	\label{for:formula1}
	{f^\wedge_{A}}(n) = \max_{D \in D_{n}}\{f_{A}(D)\}
\end{equation}
по аналогии через ${f^\vee_{A}}(n)$ будем обозначать трудоемкость в лучшем случае, как 
трудоемкость с наименьшим количеством операций на всех входах длины n:
\begin{equation}
	\label{for:formula2}
	{f^\vee_{A}}(n) = \min_{D \in D_{n}}\{f_{A}(D)\}
\end{equation}
\section{Трудоёмкость в среднем}

Трудоемкость алгоритма в среднем --- это среднее количество операций, задаваемых алгоритмом A на входах длины n, где усреднение берется по всем $D \in D_{n}$.
Введем для трудоемкости в среднем обозначение $\overline{f_{A}}(n)$, тогда
\begin{equation}
	\label{for:formula3}
	\overline{f_{A}}(n) = \sum_{D \in D_{n}} p(D) \cdot f_{A}(D),
\end{equation}
где $p(D)$ есть частотная встречаемость входа D для анализируемой области применения алгоритма. 

\section{Алгоритм полного перебора}
Алгоритм полного перебора --- метод решения задачи, при котором по очереди рассматриваются все возможные варианты решения. 
В случае реализации алгоритма в рамках данной работы будут последовательно перебираться каждый элемент до тех пор, пока не будет найден нужный.