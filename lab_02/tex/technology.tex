\chapter{Технологическая часть}

В данном разделе будут указаны средства реализации, будут представлены реализации алгоритмов, а также функциональные тесты.

\section{Требования к программному обеспечению}
Программе, реализующей данные алгоритмы, на вход будут подаваться две матрицы. Выходными данными такой программы должна быть матрица - результат умножения введенных. Программа должна работать в рамках следующих ограничений: 

\begin{itemize}
	\item элементы матрицы - целые числа;
	\item количество столбцов одной матрицы должно совпадать с количеством строк второй матрицы;
	\item должно быть выдано сообщение об ошибке при вводе пустых матриц.
\end{itemize}

Пользователь должен иметь возможность выбора алгоритма матричного умножения и вывода результата на экран. Также должны быть реализованы сравнение алгоритмов по времени работы с выводом результатов на экран и получение графического представления результатов сравнения. Данные действия пользователь должен выполнять при помощи меню.

\section{Средства реализации}

Реализация данной лабораторной работы выполнялась при помощи языка программирования С++. Данный выбор обусловлен наличием у языка функции $clock()$ \cite {clock} измерения процессорного времени.

Визуализация графиков с помощью библиотеки $Matplotlib$ \cite {matplot}.

\section{Сведения о модулях программы}

Программа состоит из следующих модулей:

\begin{itemize}
	\item main.cpp - точка входа программы;
	\item matrix.h - заголовочный файл, содержащий объявлении функций работы с матрицами;
	\item matrix.cpp - файл, содержащий реализации функций работы с матрицами;
	\item measure.h - заголовочный файл, содержащий объявлении функций замеров времени работы и затравоченной памяти рассматриваемых алгоритмов;
	\item measure.cpp - файл, содержащий реализации функций замеров времени работы и затравоченной памяти рассматриваемых алгоритмов;
\end{itemize}

\section{Реализация алгоритмов}

Стандартный алгоритм, алгоритм Винограда и оптимизированный алгоритм Винограда умножения матриц приведены в листингах \ref{lst:standard}-\ref{lst:opt_vinograd}.

\begin{lstinputlisting}[
	caption={Стандартный алгоритм умножения матриц},
	label={lst:standard},
	linerange={61-74}
	]{../src/main/Matrix.cpp}
\end{lstinputlisting}

\begin{lstinputlisting}[
	caption={Алгоритм Винограда умножения матриц},
	label={lst:vinograd},
	linerange={76-112}
	]{../src/main/Matrix.cpp}
\end{lstinputlisting}

\begin{lstinputlisting}[
	caption={Оптимизированный алгоритм Винограда умножения матриц},
	label={lst:opt_vinograd},
	linerange={114-157}
	]{../src/main/Matrix.cpp}
\end{lstinputlisting}

\clearpage

\section{Функциональные тесты}

В таблице \ref{tbl:func_test} приведены функциональные тесты для функций, реализующих алгоритмы умножения матриц. Все тесты пройдены успешно.

\begin{table}[h]
	\begin{center}
		\begin{threeparttable}
		\captionsetup{justification=raggedright,singlelinecheck=off}
		\caption{\label{tbl:func_test} Функциональные тесты}
		\begin{tabular}{|c@{\hspace{7mm}}|c@{\hspace{7mm}}|c@{\hspace{7mm}}|c@{\hspace{7mm}}|c@{\hspace{7mm}}|c@{\hspace{7mm}}|}
			\hline
			Матрица A & Матрица B & Ожидаемый результат \\ 
			\hline
			$\begin{pmatrix}
				&
			\end{pmatrix}$ &
			$\begin{pmatrix}
				&
			\end{pmatrix}$ &
			Сообщение об ошибке \\ \hline

			$\begin{pmatrix}
				&
			\end{pmatrix}$ &
			$\begin{pmatrix}
				1 & 1\\
				1 & 1\\
				1 & 1
			\end{pmatrix}$ &
			Сообщение об ошибке \\ \hline

			$\begin{pmatrix}
				1 & 2 & 3
			\end{pmatrix}$ &
			$\begin{pmatrix}
				4 & 5 & 6
			\end{pmatrix}$ &
			Сообщение об ошибке \\ \hline

			$\begin{pmatrix}
				4 & 1 & 2 \\
				-1 & 5 & -2 \\
				-7 & 0 & 1
			\end{pmatrix}$ &
			$\begin{pmatrix}
				1 & 4 & 1 \\
				1 & 1 & 3 \\
				0 & 0 & 2
			\end{pmatrix}$ &
			$\begin{pmatrix}
				5 & 17 & 11 \\
				4 & 1 & 10 \\
				-7 & -28 & -5
			\end{pmatrix}$ \\ \hline

			$\begin{pmatrix}
				1 & 2 & 3
			\end{pmatrix}$ &
			$\begin{pmatrix}
				3 \\
				2 \\
				1
			\end{pmatrix}$ &
			$\begin{pmatrix}
				10 \\
			\end{pmatrix}$ \\ \hline

		\end{tabular}
		\end{threeparttable}
	\end{center}
	
\end{table}

\section{Вывод}

Были представлены листинги реализаций всех алгоритмов умножения матриц -- стандартного, Винограда и оптимизированного алгоритма Винограда. Также в данном разделе была приведены функциональные тесты.