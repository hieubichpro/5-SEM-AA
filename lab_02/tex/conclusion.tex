\chapter*{Заключение}
\addcontentsline{toc}{chapter}{Заключение}

В результате исследования было получено, что при размере матриц, большем 40, небходимо использовать оптимизированный алгоритм умножения матриц по Винограду, так как данный алгоритм работает быстрее стандартного алгоритма в 1.45 раза. При этом стандартный алгоритм медленнее алгоритма Винограда в 1.3 раза. Стандартный алгоритм использует меньше всего памяти. Затрата памяти алгоритма Винограда и его оптимизированной версии немного различается.

Кроме того алгоритм Винограда предпочтительно использовать для умножения матриц четных размеров, так как указанный алгоритм работает в 1.08 раза быстрее, чем на матрицах с нечетным размером. Это связано с проведением дополнительных вычислений для крайних строк и столбцов.

Цель, поставленная перед началом работы, была достигнута. В ходе лабораторной работы были решены следующие задачи:

\begin{enumerate}[label={\arabic*)}]
	\item описаны алгоритмы стандартного умножения и алгоритм Винограда;
	\item построены схемы следующих алгоритмов:
	\begin{itemize}[label=---]
		\item классический алгоритм умножения матриц;
		\item алгоритм Винограда;
		\item оптимизированный алгоритм Винограда.
	\end{itemize}
	\item создано программное обеспечение, реализующее перечисленные алгоритмы;
	\item проведен сравнительный анализ реализованных алгоритмов;
	\item подготовлен отчет о выполненной лабораторной работе.
\end{enumerate}
