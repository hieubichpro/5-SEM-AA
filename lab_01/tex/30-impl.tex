\chapter{Технологическая часть}

В данном разделе будут приведены требования к ПО, средства реализации и листинги кода.

\section{Требования к ПО}

Программа принимает две строки с учетом регистров. В качестве результата возвращается число, равное редакторскому расстоянию. Необходимо реализовать возможность сравнения по времени и по памяти.

\section{Средства реализации}

В качестве языка программирования для реализации данной лабораторной работы был выбран ЯП Python. 

Данный язык имеет все небходимые инструменты для решения поставленной задачи.

\section{Сведения о модулях программы}
Программа состоит из трех модулей:
\begin{enumerate}[label={\arabic*)}]
    \item main.py --- точка входа;
	\item algorythms.py --- хранит реализацию алгоритмов;
	\item compare.py --- хранит реализацию системы замера времени.
\end{enumerate}


\section{Листинг кода}

 В листингах \ref{lst:non_rec_l}, \ref{lst:non_rec_dl}, \ref{lst:rec_dl}, \ref{lst:rec_dl_cache} приведены реализаций алгоритмов нахождения расстояния Левенштейна и Дамерау--Левенштейна.

\begin{lstlisting}[language = Python, label=lst:non_rec_l,caption=Функция нахождения расстояния Левенштейна нерекурсивным методом]
def levenshteinDistance(A, B, output = True):
	n = len(A)
	m = len(B)
	L = createMatrix(n + 1, m + 1)

	for i in range(1, n + 1):
		for j in range(1, m + 1):
			if A[i - 1] == B[j - 1]:
				L[i][j] = L[i - 1][j - 1]
			else:
				L[i][j] = min(L[i - 1][j], 
							  L[i][j - 1],
							  L[i - 1][j - 1]
							  ) + 1

	if (output):
		printMatrix(L, A, B)

	return L[n][m]
	
\end{lstlisting}

\begin{lstlisting}[language = Python,label=lst:non_rec_dl,caption=Функция нахождения расстояния Дамерау--Левенштейна нерекурсивным методом]
def damerauLevenshteinDistance(A, B, output = True):
	n = len(A)
	m = len(B)
	L = createMatrix(n + 1, m + 1)

	for i in range(1, n + 1):
		for j in range(1, m + 1):
			if A[i - 1] == B[j - 1]:
				L[i][j] = L[i - 1][j - 1]
			else:
				L[i][j] = min(
							  L[i - 1][j],
							  L[i][j - 1],
							  L[i - 1][j - 1]
							  ) + 1

			if (i > 1 and j > 1 and A[i - 1] == B[j - 2] and A[i - 2] == B[j - 1]):
				L[i][j] = min(L[i][j], L[i - 2][j - 2] + 1)

	if (output):
		printMatrix(L, A, B)

	return L[n][m]
\end{lstlisting}

\begin{lstlisting}[language = Python,label=lst:rec_dl,caption=Функция нахождения расстояния Дамерау--Левенштейна с использованием рекурсии]
def damerauLevenshteinDistanceRecursive(A, B, output = True):
	n = len(A)
	m = len(B)

	if ((n == 0) or (m == 0)):
		return abs(n - m)

	D = min(
			damerauLevenshteinDistanceRecursive(A[:-1], B) + 1,
			damerauLevenshteinDistanceRecursive(A, B[:-1]) + 1,
			damerauLevenshteinDistanceRecursive(A[:-1], B[:-1]) + (A[-1] != B[-1])
			)

	if (n > 1 and m > 1 and A[-1] == B[-2] and A[-2] == B[-1]):
		D = min(D, damerauLevenshteinDistanceRecursive(A[:-2], B[:-2]) + 1)

	return D

\end{lstlisting}

\begin{lstlisting}[language = Python,label=lst:rec_dl_cache,caption=Функция нахождения расстояния Дамерау--Левенштейна рекурсивным методом с использованием кеша]
def recursiveWithCache(A, B, n, m, L):
	if (L[n][m] != -1):
		return L[n][m]

	if n == 0 or m == 0:
		L[n][m] = abs(n - m)
		return L[n][m]

	L[n][m] = min(
				recursiveWithCache(A, B, n - 1, m, L) + 1,
				recursiveWithCache(A, B, n, m - 1, L) + 1,
				recursiveWithCache(A, B, n - 1, m - 1, L) + (A[n - 1] != B[m - 1])
				)
	if (n > 1 and m > 1 and
	A[n - 1] == B[m - 2] and
	A[n - 2] == B[m - 1]):

		L[n][m] = min(L[n][m], recursiveWithCache(A, B, n - 2, m - 2, L) + 1)

	return L[n][m]
	
def damerauLevenshteinDistanceRecurciveCache(A, B, output = True):
	n = len(A)
	m = len(B)
	L = createMatrix(n + 1, m + 1)

	for i in range(n + 1):
		for j in range(m + 1):
			L[i][j] = -1

	recursiveWithCache(A, B, n, m, L)

	if (output):
		printMatrix(L, A, B)

	return L[n][m]

\end{lstlisting}

\section{Функциональные тесты}
В таблице \ref{tabular:functional_test} приведены функциональные тесты для алгоритмов вычисления расстояния Левенштейна и Дамерау-Левенштейна. Все тесты пройдены.

\begin{table}[h]
    \begin{center}
    \begin{threeparttable}
    \captionsetup{justification=raggedright,singlelinecheck=off}	\caption{\label{tabular:functional_test} Функциональные тесты}
		\begin{tabular}{|l|l|l|l|l|}
			\hline
			&  \multicolumn{2}{c|}{Входные данные}& \multicolumn{2}{c|}{Ожидаемый результат} \\
			\hline
			№&Строка 1&Строка 2&Левенштейн&Дамерау-Л. \\
			\hline
			1&"пустая строка"&"пустая строка"&0&0 \\
			\hline
			2&"пустая строка"&x&1&1 \\
			\hline
			3&bmstu&"пустая строка"&5&5 \\
			\hline
			4&"пустая строка"&iu7&3&3 \\
			\hline
			5&lll&lll&0&0 \\
			\hline
			6&something&sommer&6&6 \\
			\hline
			7&united&unique&3&2 \\
			\hline
			8&summer&service&3&3 \\
			\hline
            9&abcdef&acbdef&2&1\\
            \hline
		\end{tabular}
	\end{threeparttable}
	\end{center}
\end{table}

\section*{Вывод}

Были разработаны и протестированы алгоритмы нахождения расстояния Левенштейна нерекурсивно, нахождения расстояния Дамерау -- Левенштейна нерекурсивно, рекурсивно, а также рекурсивно с кешированием.
