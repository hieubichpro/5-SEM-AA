\chapter*{Введение}
\addcontentsline{toc}{chapter}{Введение}

Целью данной лабораторной работы является описание, реализация и исследование алгоритмов нахождения расстояний Левенштейна и Дамерау-Левенштейна.

\bigskip

\textbf{Расстояние Левенштейна} -- это минимальное количество односимвольных операций (вставки, удаления, замены), необходимых для превращения одной последовательности символов в другую.

\bigskip

\textbf{Расстояние Дамерау-Левенштейна} -- модификация расстояния Левештейна. Это минимальное количество операций вставки, удаления, замены и транспозиции (перестановки двух соседних символов), необходимых для перевода одной строки в другую.

\bigskip

Расстояния Левенштейна и  Дамерау-Левенштейна широко применяются для решения задач компьютерной лингвистики (исправление ошибок в слове, автоматическое распознавание отсканированного текста или речи), биоинформатики (для сравнения генов) и других.


\bigskip

В рамках выполнения лабораторной работы необходимо решить следующие задачи:

\begin{itemize}
	\item[---] описать расстояния Левенштена и Дамерау-Левенштейна;
	\item[---] построить схемы алгоритмов следующих методов: нерекурсивный метод поиска расстояния Левенштейна, нерекурсивный метод поиска Дамерау-Левенштейна, рекурсивный метод поиска Дамерау-Левенштейна, рекурсивный с кешированием метод поиска Дамерау-Левенштейна;
	\item[---] создать ПО, реализующее перечисленные выше алгоритмы;
	\item[---] сравнить алгоритмы определения расстояния между строками по затрачиваемым ресурсам (времени и памяти);
	\item[---] описать и обосновать полученные результаты в отчете о выполненной лабораторной работе.
\end{itemize}
