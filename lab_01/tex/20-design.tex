\chapter{Конструкторская часть}
В этом разделе будут приведены требования к вводу и программе, а также схемы алгоритмов нахождения расстояний Левенштейна и Дамерау-Левенштейна.

\section{Требования к вводу}
На вход подаются две строки, причем буквы верхнего и нижнего регистров считаются различными.

\section{Требования к программе}
При вводе двух пустых строк программа не должна аварийно завершатся. Вывод программы - число (расстояние Левенштейна или Дамерау-Левенштейна), для алгоритмов, использующих матрицу для расчёта требуется вывести заполненную матрицу.

\section{Разработка алгоритма поиска расстояния Левенштейна}

На рисунке \ref{img:l} приведена схема нерекурсивного алгоритма нахождения расстояния Левенштейна.

\img{190mm}{l}{Схема нерекурсивного алгоритма нахождения расстояния Левенштейна}

\newpage

\section{Разработка алгоритмов поиска \\ расстояния Дамерау-Левенштейна}

На рисунке \ref{img:dl} приведена схема нерекурсивного алгоритма нахождения расстояния Дамерау-Левенштейна.

\img{190mm}{dl}{Схема нерекурсивного алгоритма нахождения расстояния Дамерау-Левенштейна}

\newpage

На рисунке \ref{img:rdl} приведена схема рекурсивного алгоритма нахождения расстояния Дамерау-Левенштейна.

\img{175mm}{rdl}{Схема рекурсивного алгоритма нахождения расстояния Дамерау-Левенштейна}

\newpage

На рисунке \ref{img:rdl_cache} приведена схема рекурсивного алгоритма нахождения расстояния Дамерау-Левенштейна с использованием кеша в виде матрицы.

\img{180mm}{rdl_cache}{Схема рекурсивного алгоритма нахождения расстояния Дамерау-Левенштейна с использованием кеша в виде матрицы}

\newpage

\section*{Вывод}

Перечислены требования к вводу и программе, а также были построены схемы требуемых алгоритмов.






